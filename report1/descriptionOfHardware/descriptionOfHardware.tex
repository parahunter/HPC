 - stuff from lscpu
 - compiler and libraries used. version numbers to

\subsecion{Hardware} 
 
To perform the test, it has been used the DTU HPC facility, in particular a Intel Xeon X5550 @ 2.67 GHz.

More technical specification are listed below, in particular please note the cache size, 32 KB L1 and 256 KB L2, and the current CPU speed during the tests, reduced at 1.6 GHz.

DTU provides also some AMD Opteron 6168 machines (technical specifications are listed below as well), but if not different specified, the Intel Xeon X5550 is used.
 
\begin{tabular}{ l l }
\hline
Architecture:          & x86_64\\ \hline
CPU op-mode(s):        & 32-bit, 64-bit\\ \hline
Byte Order:            & Little Endian\\ \hline
CPU(s):                & 8\\ \hline
On-line CPU(s) list:   & 0-7\\ \hline
Thread(s) per core:    & 1\\ \hline
Core(s) per socket:    & 4\\ \hline
CPU socket(s):         & 2\\ \hline
NUMA node(s):          & 2\\ \hline
Vendor ID:             & GenuineIntel\\ \hline
CPU family:            & 6\\ \hline
Model:                 & 26\\ \hline
Stepping:              & 5\\ \hline
CPU MHz:               & 1600.000\\ \hline
BogoMIPS:              & 5332.58\\ \hline
Virtualization:        & VT-x\\ \hline
L1d cache:             & 32K\\ \hline
L1i cache:             & 32K\\ \hline
L2 cache:              & 256K\\ \hline
L3 cache:              & 8192K\\ \hline
NUMA node0 CPU(s):     & 4-7\\ \hline
NUMA node1 CPU(s):     & 0-3\\ \hline
\end{tabular}

From now on we’ll refer to the Intel Xeon X5550 also as ‘the Intel’ or ‘the Xeon’, while we’ll refer to the AMD Opteron 6168 as ‘the AMD’.

\subsecion{Compiler} 

The compiler that has been used is Sun Compiler 5.12 

\begin{lstlisting}
suncc -g -fast -fPIC  -c -o lib.o lib.c
\end{lstlisting}

Please note that, although DTU provides some AMD machines as well, it is necessary to compile in the Intel machine, as the compiler is affected by some bugs. If it runs on AMD cpu, it will cause ‘Segmentation Fault’ on the final application.

Additionally it has been used the Oracle Solaris Studio Performances Analyzer 7.9 to profile the code. This software, that will be referred as “the Analyzer” from now on, it has been run over the AMD platform as it has a better support for hardware counters.
